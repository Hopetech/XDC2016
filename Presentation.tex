%%%%%%%%%%%%%%%%%%%%%%%%%%%%%%%%%%%%%%%%%
% Beamer Presentation
% LaTeX Template
% Version 1.0 (10/11/12)
%
% This template has been downloaded from:
% http://www.LaTeXTemplates.com
%
% License:
% CC BY-NC-SA 3.0 (http://creativecommons.org/licenses/by-nc-sa/3.0/)
%
%%%%%%%%%%%%%%%%%%%%%%%%%%%%%%%%%%%%%%%%%

%----------------------------------------------------------------------------------------
%	PACKAGES AND THEMES
%----------------------------------------------------------------------------------------

\documentclass{beamer}

\mode<presentation> {

% The Beamer class comes with a number of default slide themes
% which change the colors and layouts of slides. Below this is a list
% of all the themes, uncomment each in turn to see what they look like.

%\usetheme{default}
%\usetheme{AnnArbor}
%\usetheme{Antibes}
%\usetheme{Bergen}
%\usetheme{Berkeley}
%\usetheme{Berlin}
%\usetheme{Boadilla}
%\usetheme{CambridgeUS}
%\usetheme{Copenhagen}
%\usetheme{Darmstadt}
%\usetheme{Dresden}
\usetheme{Frankfurt}
%\usetheme{Goettingen}
%\usetheme{Hannover}
%\usetheme{Ilmenau}
%\usetheme{JuanLesPins}
%\usetheme{Luebeck}
%\usetheme{Madrid}
%\usetheme{Malmoe}
%\usetheme{Marburg}
%\usetheme{Montpellier}
%\usetheme{PaloAlto}
%\usetheme{Pittsburgh}
%\usetheme{Rochester}
%\usetheme{Singapore}
%\usetheme{Szeged}
%\usetheme{Warsaw}

% As well as themes, the Beamer class has a number of color themes
% for any slide theme. Uncomment each of these in turn to see how it
% changes the colors of your current slide theme.

%\usecolortheme{albatross}
%\usecolortheme{beaver}
%\usecolortheme{beetle}
%\usecolortheme{crane}
%\usecolortheme{dolphin}
%\usecolortheme{dove}
%\usecolortheme{fly}
%\usecolortheme{lily}
%\usecolortheme{orchid}
%\usecolortheme{rose}
%\usecolortheme{seagull}
%\usecolortheme{seahorse}
%\usecolortheme{whale}
%\usecolortheme{wolverine}

%\setbeamertemplate{footline} % To remove the footer line in all slides uncomment this line
%\setbeamertemplate{footline}[page number] % To replace the footer line in all slides with a simple slide count uncomment this line

%\setbeamertemplate{navigation symbols}{} % To remove the navigation symbols from the bottom of all slides uncomment this line
}

\usepackage{graphicx} % Allows including images
\usepackage{booktabs} % Allows the use of \toprule, \midrule and \bottomrule in tables

%----------------------------------------------------------------------------------------
%	TITLE PAGE
%----------------------------------------------------------------------------------------

\title[Short title]{WIP : Implementation of a double floating point library in GLSL 1.30}

\author{Elie TOURNIER}
\institute[GSoC]
{
Google Summer of Code 2016

\medskip
\textit{tournier.elie@gmail.com}
}
\date{\today}

\begin{document}

\begin{frame}
\titlepage % Print the title page as the first slide
\end{frame}

%\begin{frame}
%\frametitle{Overview} % Table of contents slide, comment this block out to remove it
%\tableofcontents % Throughout your presentation, if you choose to use \section{} and \subsection{} commands, these will automatically be printed on this slide as an overview of your presentation
%\end{frame}

%----------------------------------------------------------------------------------------
%	PRESENTATION SLIDES
%----------------------------------------------------------------------------------------

%------------------------------------------------
\section{About me} % Sections can be created in order to organize your presentation into discrete blocks, all sections and subsections are automatically printed in the table of contents as an overview of the talk
%------------------------------------------------

\begin{frame}
\frametitle{Who am I?}
\begin{itemize}
\item Elie Tournier
\item Graduate Software and Image Processing Engineer
\item Google Summer of Code 2016 Student
\item Available for hire !
\end{itemize}
\end{frame}

%------------------------------------------------
\section{The project}
%------------------------------------------------

\begin{frame}
\frametitle{Goal}
\begin{itemize}
\item Create GL\_ARB\_gpu\_shader\_fp64 for GPU before OpenGL 4.0 compatibility.
\item So we need double precision support.
\end{itemize}
\end{frame}

%------------------------------------------------

\begin{frame}
\frametitle{Constraints}
\begin{itemize}
\item The license should be compatible with Mesa.
\item IEEE 754 compliant.
\end{itemize}
\end{frame}

%------------------------------------------------
\section{The work}
%------------------------------------------------

\begin{frame}
\frametitle{Choose a CPU lib}
\begin{itemize}
\item We don't want to reinventing the wheel.
\item Convert a CPU library to a GPU one.
\item Berkeley SoftFloat from John R. Hauser
\end{itemize}
\end{frame}

%------------------------------------------------
\section{The dev}
%------------------------------------------------

\begin{frame}
\frametitle{Dev environment}
\begin{itemize}
\item Use Shader\_runner from Piglit
\item Store fp64 in uvec2().
\end{itemize}
\end{frame}

%------------------------------------------------

\begin{frame}[fragile] % Need to use the fragile option when verbatim is used in the slide
\frametitle{Code Example}
\begin{example}[Code softfloat]
\begin{verbatim}
int main(){
  return 0;
}
\end{verbatim}
\end{example}

\frametitle{Verbatim}
\begin{example}[Code libSoftFloat]
\begin{verbatim}
int main(){
  return 0;
}
\end{verbatim}
\end{example}

\end{frame}

%------------------------------------------------

\begin{frame}
\frametitle{References}
\footnotesize{
\begin{thebibliography}{99} % Beamer does not support BibTeX so references must be inserted manually as below
\bibitem[Hauser, 2015]{p1} John R. Hauser (2015)
\newblock Berkeley SoftFloat
\newblock U.C. Berkeley.
\newblock http://www.jhauser.us/arithmetic/SoftFloat.html

\bibitem[Tournier, 2016]{p1} Elie Tournier (2016)
\newblock libSoftFloat
\newblock https://github.com/Hopetech/libSoftFloat

\end{thebibliography}
}
\end{frame}

%------------------------------------------------

\begin{frame}
\Huge{\centerline{Thanks.}}
\end{frame}

%----------------------------------------------------------------------------------------

\end{document}